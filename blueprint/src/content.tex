% In this file you should put the actual content of the blueprint.
% It will be used both by the web and the print version.
% It should *not* include the \begin{document}
%
% If you want to split the blueprint content into several files then
% the current file can be a simple sequence of \input. Otherwise It
% can start with a \section or \chapter for instance.

\chapter{Free Group of Rotations}

\section{Selection of Rotations and Decomposition of the Majority of the Sphere}

We specifically choose two rotations of the sphere, with which we will construct the decomposition in the theorem.

\begin{definition}\label{def:spec_rot}
Our rotation matrices are:
\end{definition}

\begin{lemma}[Invertibility of A and B] \label{lemma:a_b_invertible}
It holds that $\det A \neq 0$ and $\det B \neq 0$, and thus $A$ and $B$ are invertible.
\end{lemma}
\begin{proof} \uses{def:spec_rot}
This follows by direct calculation.
\end{proof}

$A$ and $B$ therefore generate a subgroup of the group of invertible $3 \times 3$ matrices.

\begin{definition}\label{def:spec_rot_generated} \uses{lemma:a_b_invertible}
Let $G$ denote the subgroup generated by $A$ and $B$.
\end{definition}

\begin{lemma}\label{lem:adjugate_fin_three}
The adjugate of a $3 \times 3$ matrix can be represented in a specific form...
\end{lemma}
\begin{proof}
\end{proof}

\begin{lemma}[Concrete Representation of the Rotations]\label{lem:repr_of_rot_res}
If $\rho : \mathbb{R}^3 \rightarrow \mathbb{R}^3$ is an expression in $G$ of length $n$ in reduced form, then $\rho(0,1,0)$ is of the following form, where $a$, $b$, and $c$ are integers: $\rho(0,1,0) = \frac{1}{3^n}(a\sqrt{2}, b, c\sqrt{2})$.
\end{lemma}
\begin{proof} \uses{def:spec_rot_generated,lem:adjugate_fin_three}
This assertion follows from the generator matrices and by explicitly multiplying a reduced word by $(0,1,0)$.
\end{proof}

With this, we can show that this subgroup of rotations is a free group with two generators.

\begin{definition}\label{def:free_grp}
A free group $G$ is a group in which two words on a specific generating set are different, except when their equality follows from the group axioms.
\end{definition}

\begin{theorem}\label{thm:free_grp_of_rot}
The subgroup $G$ generated by our specific rotations from Definition \ref{def:spec_rot} is a free group.
\end{theorem}
\begin{proof}\uses{lem:repr_of_rot_res,def:free_grp}
This follows directly.
\end{proof}
\chapter{Doubling a Sphere}

\section{Sphere}
\begin{definition}[Unit Sphere without Center]\label{def:sphere_without_center}
Let $L = \{(x,y,z) : x^2 + y^2 + z^2 \leq 1\}$ be the unit sphere. We define $L' = L \setminus \{(0,0,0)\}$ as the unit sphere without its center.
\end{definition}

\begin{definition}[Orbit] \label{def:orbit} \uses{def:spec_rot_generated}
Two points $a$ and $b$ belong to the same orbit if and only if there exists a $\rho$ in $G$ such that $\rho(a) = b$.
\end{definition}

\begin{lemma}[Countability of All Orbits] \label{lemma:all_orbits_countable}
The set of all orbits is countable.
\end{lemma}
\begin{proof}
    \uses{def:orbit}
\end{proof}

\begin{lemma}[Representative Points] \label{theorem:rep_points}
We can select a representative point from each orbit.
\end{lemma}
\begin{proof} \uses{def:orbit}
This follows directly from the Axiom of Choice.
\end{proof}

\begin{definition}[Set of All Representatives] \label{def:set_rep_points} \uses{theorem:rep_points}
Let $M$ denote the set of all chosen representative points.
\end{definition}

We can now reach every point in $L'$ by applying a rotation from $G$ to a specific element in $M$.

\section{Duplicating a Part of the Unit Sphere}

We desire that each point in $L'$ is reached by only one rotation in $G$. Therefore, we decompose $L'$ according to the rotations that map to a point. A point that is mapped by multiple rotations may belong to multiple sets.

\begin{definition}[Fixed Points] \label{def:fixed_points}
Let $Y$ be a set and $f: Y \rightarrow Y$ a function. A point $y \in Y$ is called a fixed point if it satisfies $f(y) = y$.
\end{definition}

\begin{definition}[Set of All Fixed Points] \label{def:set_fixed_points} \uses{def:fixed_points,def:sphere_without_center}
Let $D$ denote the set of all points in $L'$ that are fixed points of the rotations in $G$.
\end{definition}

\begin{lemma}[Countability of G] \label{lemma:G_countable}
$G$ is countable.
\end{lemma}
\begin{proof} \uses{def:spec_rot_generated, thm:free_grp_of_rot}
\end{proof}

\begin{definition}[Exactly One Rotation Axis] \label{lemma:one_rot_axis}
Each rotation in $G$ has exactly one rotation axis.
\end{definition}

\begin{lemma}[Countability of Rotation Axes] \label{lemma:count_rot_axes}
The rotation axes lie on countably many lines.
\end{lemma}
\begin{proof} \uses{def:set_fixed_points,lemma:G_countable,lemma:one_rot_axis, lemma:count_set_rep_points}
\end{proof}

Therefore, almost every point in $L'$ can be reached by a specific rotation. We now consider a decomposition of $L' \setminus D$ and will address the fixed points later.

\begin{definition}[Union X] \label{def:union_x} \uses{def:set_rep_points,def:spec_rot}
$X = \bigcup\limits_{k=1}^{\infty} A^{-k} M$. Thus, $X$ is the set of all elements of $M$ that consist solely of rotations by $A^{-1}$.
\end{definition}

\begin{definition}[Decomposition into Sets] \label{def:decomposition_L_D} \uses{def:set_rep_points,def:union_x,def:spec_rot}
\begin{align*}
P_1 &= S(A)M \cup M \cup X \\
P_2 &= S(A^{-1})M \setminus X \\
P_3 &= S(B)M \\
P_4 &= S(B^{-1})M
\end{align*}
\end{definition}

\begin{lemma}[Union of the Decomposition] \label{lemma:union_decomposition}
$L' \setminus D = P_1 \cup P_2 \cup P_3 \cup P_4$
\end{lemma}
\begin{proof} \uses{def:decomposition_L_D,def:set_fixed_points,def:sphere_without_center}
\end{proof}

\begin{lemma}[Rotation of Decomposed Sets] \label{lemma:rot_decomposed_sets}
\begin{align*}
A P_2 &= P_2 \cup P_3 \cup P_4 \\
B P_4 &= P_1 \cup P_2 \cup P_4
\end{align*}
\end{lemma}
\begin{proof} \uses{def:decomposition_L_D,def:spec_rot}
\end{proof}

\begin{lemma}[Doubling of L' \ D] \label{lemma:doubling_L_D}
\begin{align*}
L' \setminus D &= P_1 \cup A P_2 \\
L' \setminus D &= P_3 \cup B P_4
\end{align*}
\end{lemma}
\begin{proof} \uses{lemma:rot_decomposed_sets,lemma:union_decomposition}
\end{proof}

\begin{lemma}[Countability of the Set of Representative Points] \label{lemma:count_set_rep_points}
The set of representative points is countable.
\uses{def:set_rep_points,lemma:all_orbits_countable}
\end{lemma}
\begin{proof}
\end{proof}

We now have a decomposition that allows us to duplicate the sphere, excluding its center and the points on the rotation axes.

\section{Fixed Points and the Center}

\begin{definition}[Equidecomposability] \label{def:equidecomposable}
Two sets $C$ and $D$ are called equidecomposable if $C$ can be partitioned into finitely many pieces, which can be rearranged through rotations and translations to form $D$.
\end{definition}

\begin{lemma}[Equidecomposability of $L' \setminus D$ and $L'$] \label{lem:equidecomposability}
$L' \setminus D$ and $L'$ are equidecomposable.
\end{lemma}
\begin{proof}\uses{def:equidecomposable,lemma:countable_rotation_axes}
Since the points in $D$ lie on countably many axes, we can find a line $l$ through the origin that does not intersect $D$.
Furthermore, there exists an angle $\theta$ such that a rotation by $\theta$ around $l$ does not map any point in $D$ onto another point in $D$.
Define $E = D \cup \rho(D) \cup \rho^2(D) \cup \rho^3(D) \cup \dots$. It follows immediately that
$L' = E \cup (L' \setminus E)$, which is equidecomposable with $\rho(E) \cup (L' \setminus E)$.
From the definition of $E$, we obtain $\rho(E) = E \setminus D$, so that we get
$\rho(E) \cup (L' \setminus E) = (E \setminus D) \cup (L' \setminus E) = L' \setminus D$.
Thus, it follows that $L'$ is equidecomposable with $L' \setminus D$.
\end{proof}

Now that we know that $L' \setminus D$ is equidecomposable with $L'$, the fixed points of the rotations in $G$ are no longer an issue.
Hence, we now focus on the center.

\begin{lemma}[Pi and Square Root of 2 Have No Common Multiple] \label{lemma:ncm_pi_sqrt_2}
$\nexists p,q\in \mathbb{Z}$ such that $\sqrt{2} \cdot \frac{p}{q} = \pi$

\end{lemma}
\begin{proof}
Assume that there exist $p,q\in \mathbb{Z}$ such that $\sqrt{2} \cdot \frac{p}{q} = \pi$. By the definition of $\cos^{-1}$ on the interval $[0,2\pi)$, we have $\cos^{-1}(-1) = \pi$ and thus
\[
\cos^{-1}(-1) = \sqrt{2} \cdot \frac{p}{q}.
\]
This is equivalent to $-1 = \cos(\sqrt{2} \cdot \frac{p}{q})$. Using the Taylor series expansion of cosine, we would obtain:
\[
\sum_{n=0}^{\infty} \frac{(-1)^n}{(2n)!} (2 \cdot \frac{p^2}{q^2})^n = -1.
\]
This contradicts the properties of rational approximations, proving the claim.
\end{proof}

\begin{lemma}[Equidecomposability of a Circle] \label{lemma:equidecomposable_circle}
A circle is equidecomposable with a circle minus a single point.
\end{lemma}
\begin{proof} \uses{def:equidecomposable,lemma:ncm_pi_sqrt_2}

We consider the unit circle $S^1 = \{(x,y) : x^2 + y^2 = 1\}$ minus the point $(1,0)$. We use the unit circle to simplify notation, but the following argument can be applied to any circle.
Define $A = \{(\cos n, \sin n) : n \in \mathbb{Z}\}$. Since $\pi$ is irrational, we have $(\cos n, \sin n) \neq (\cos m, \sin m)$ for all distinct $n, m \in \mathbb{Z}$.
Thus, $A$ is countably infinite and does not contain $(1,0)$. Let $B = (S^1 \setminus \{(1,0)\}) \setminus A$.
Now, rotate $A$ about the origin by one unit. Denote the rotated set by $A'$. This rotation maps $(\cos 1, \sin 1)$ to $(1,0)$,
which is exactly the missing point. Since every point originally in $A$ is still present in $A'$, we obtain $A' = A \cup \{(1,0)\}$,
and thus $S^1 \setminus \{(1,0)\} = A \cup B$ is equidecomposable with $A' \cup B = S^1$.
\end{proof}

\begin{lemma}[Equidecomposability of a Subset]
\label{lemma:equidecomposable_subset}

\end{lemma}
\begin{proof}
\uses{def:equidecomposable}

\end{proof}

\begin{theorem}[Equidecomposability of a Sphere] \label{theorem:equidecomposable_sphere}
A sphere minus its center is equidecomposable with the complete sphere.
\end{theorem}
\begin{proof} \uses{lemma:equidecomposable_circle, lemma:equidecomposable_subset}
Constructing a circle within the sphere that includes the center of the sphere, together with Lemma \ref{lemma:equidecomposable_circle},
shows that the sphere minus its center is equidecomposable with the complete sphere.
\end{proof}

\section{The Final Proof}

Having gathered all the necessary details, we can now establish the full paradox.
\begin{theorem}[Banach-Tarski]\label{thm:BanachTarski}
A sphere is equidecomposable with two copies of itself.
\end{theorem}
\begin{proof} \uses{lem:equidecomposability,theorem:equidecomposable_sphere,lemma:doubling_L_D}
In the second subsection, we showed that the sphere minus its center and the points on the rotation axes is equidecomposable with two copies of itself.
Using Lemma \ref{lem:equidecomposability}, it follows that the sphere minus its center is equidecomposable with two copies of itself.
Finally, by Theorem \ref{theorem:equidecomposable_sphere}, the complete sphere is equidecomposable with two copies of itself.
\end{proof}
